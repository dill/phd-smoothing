% Speeding up MDS
% David Lawrence Miller
% d.l.miller@bath.ac.uk
  
\documentclass[a4paper,10pt]{article}
\setlength{\textheight}{22.5cm}
\setlength{\textwidth}{6.47in}
\setlength{\oddsidemargin}{-1mm}
\setlength{\topmargin}{0.1cm}
\setlength{\evensidemargin}{-5mm} 
 
% Load some packages
\usepackage{times, amsmath, amssymb, amsfonts, url, natbib, bm, rotating}
 
\usepackage{multirow}
\usepackage{graphicx}
\usepackage{rotating}

% top matter
\title{Improvements to the MDS+TPRS approach}
\author{David Lawrence Miller\\Mathematical Sciences\\University of Bath\\\texttt{d.l.miller@bath.ac.uk}}
 
% Shortcuts
% Probability
\newcommand{\prob}[1]{\mathbb{P}\left[ #1 \right]}
% Schwarz-Christoffel
\newcommand{\sch}{Schwarz-Christoffel }
% fprime
\newcommand{\fprime}{f^\prime(z)}
% figure reference command
\newcommand{\fig}[1]{\emph{fig.} \ref{#1}}
% Figure reference command
\newcommand{\Fig}[1]{\emph{Fig.} \ref{#1}}
% table reference command
\newcommand{\tabref}[1]{\emph{table} \ref{#1}}
% Table reference command
\newcommand{\Tabref}[1]{\emph{Table} \ref{#1}}
% equation reference command
\newcommand{\eqn}[1]{(\ref{#1})}
% phi inverse
\newcommand{\phiinv}{\phi^{-1}}
% use other phi
\renewcommand{\phi}{\varphi}
%transpose
\newcommand{\tr}[1]{#1^{\text{T}}}
% diagonal
\newcommand{\diag}{\text{diag}}
% call \times \cross
\newcommand{\cross}{\times}


\begin{document}
 
% The abstract
%\begin{abstract}
%Here.
%\end{abstract}
 
 
% New theorem for theorems
\newtheorem{thm}{Theorem}[section]
 
%New theorem for definitions
\newtheorem{defn}{Definition}[section]
 
\maketitle

\section{Introduction}

Clearly need to speed up. Also make things more accurate.

\section{Calculating MDS by Lanczos iteration}

This should be simple enough, just re-write \texttt{cmdscale} so that only the first $k$ eigenvalues are computed, rather than finding $n$, returning $k$ of them and discarding the rest.

\section{Partial path calculation}

Notation is as in algorithm for the path calculation. Taking points $p_i$ and $p_j$ in the set of points in the domain that we wish to find the shortest paths for and drawing a path between them, finding within-area distance with respect to the boundary of $\Gamma$.

This is a modification of the the original algorithm, the routines INIT, DELETE, ALTER, ITER are identical.

\begin{enumerate}
 \item Begin by creating a sparse grid within $\Gamma$ and calculate the 


\end{enumerate}

\item First calculate the Euclidean distances between all unique $p_i$ and $p_j$ where $i=1,\dots, N$ and $j=1,\dots, N$ if possible (ie. if the straight line path between them lies within the domain.) Store these in two identical symmetric matrices, $D$ and $E$ (since the path from  $p_i$ to $p_j$ is the same as  $p_j$ to $p_i$.) $D$ is used to store the lengths of all paths and is updated, $E$ is not modified and just contains those paths which are Euclidean. We also store a series of previous paths $\mathcal{P}_1,\ldots, \mathcal{P}_M$ (where $M$ is at most $N-1$) which are updated through the algorithm.

\item For each unique pairing of $p_i$ and $p_j$ calculate the path using \textbf{one} of the following options.

\begin{enumerate}

\item If there has been a previous path calculated (ie. $\mathcal{P}_m$ exists for at least one $m$) and $p_i$ is the same as either end point of a previous path ($p_i=p_{k}$ for some $k$ where $p_k$ and $p_l$ are the end points of $\mathcal{P}_m$ for some $m$) then run INIT using $p_j$ and $p_l$. Take the resulting path ($\mathcal{P}_\text{INIT}=(p_l, \dots, p_j)$) and add it to the $p_l$ end of the matching previous path, $\mathcal{P}_m$, omitting one of the duplicate $p_l$. The new path is then $\mathcal{P}_*=(p_k, \dots, p_l, \dots, p_j)$. Now continue looping over the DELETE and ALTER steps until convergence as before with $\mathcal{P}_*$ as input. Save the optimal path in $\mathcal{P}_j$ and its length in $D_{ij}$.

\item If the above step was not chosen then run it again but with $p_i$ in $p_j$s place and vice versa, ie. run the step again but checking to see if $p_j =p_{k}$ for some $k$ where $p_k$ is an end point of $\mathcal{P}_m$ for some $m$.

\item If neither of (a) nor (b) have been used, then use the original algorithm to calculate the path between $p_i$ and $p_j$. Store the path in $\mathcal{P}_{j}$ and calculate and store its length in $D_{ij}$.

\end{enumerate}
\end{enumerate}

This might work well for a grid but for real data we probably want to order the data (top left to bottom right?) before running the algorithm. Some order is probably better than none.

Faster if instead of searching 0 to latest path, rather latest to 0. Since newer paths will be ``closer'' ?


\section{Clever grids}

Initial grid calculations could be done on a much less dense grid, making things much faster. Possibly just shrinking the polygon and using points on the edges.

\section{Dynamic $\lambda$}



\section{Conclusions from last time}

Although these results are encouraging, the inability of the smooth to capture all the features in the right side of the domain when using an isotropic smoother is disappointing. Proposed further work is to follow the lead from \cite{wood2000}, which shows that given some transform of a variable, $y$ say, such that $y_i^\prime=y_i/k$, then $f(x,y^\prime k)$ will give the same fit as $f(x,y)$ (ie. the fit will be the same under the new coordinates) but the penalty will change to:
\begin{equation}
\int\int_\Omega \Big( \frac{\partial^2 f}{\partial x^2} \Big) + 2k\Big( \frac{\partial^2 f}{\partial x \partial y} \Big) + k^3\Big( \frac{\partial^2 f}{\partial y^2} \Big) \text{d}x \text{d}y,
\label{adjustedintegral}
\end{equation}
from:
\begin{equation*}
\int\int_\Omega \Big( \frac{\partial^2 f}{\partial x^2} \Big) + 2\Big( \frac{\partial^2 f}{\partial x \partial y} \Big) + \Big( \frac{\partial^2 f}{\partial y^2} \Big) \text{d}x \text{d}y.
\end{equation*}
In the case of the MDS, the integral would need to be split over a grid such that each cell captured the distortion in space over the area of the cell. Mathematically, $\Omega$ would be split into a subsets, $\omega_i$ (where $\bigcup_{\forall j} \omega_j = \Omega$) and (\ref{adjustedintegral}) becomes:
\begin{equation}
\sum_{\forall j} \int\int_{\omega_j} l_j^3 \Big( \frac{\partial^2 f}{\partial x^2} \Big) + 2k_jl_j\Big( \frac{\partial^2 f}{\partial x \partial y} \Big) + k_j^3\Big( \frac{\partial^2 f}{\partial y^2} \Big) \text{d}x \text{d}y,
\end{equation}
where $k_j$ is the scaling in the $y$ direction and $l_j$ is the scaling in the $x$ direction for the $j^{\text{th}}$ cell. We assume here that $f(x,y)$ is ``nice''.

Using this penalty may allow for the distortions in space to be taken account of and therefore get around the problem of the details being smoothed over, as seen above. It will be easiest to first build this using a tensor product of $P$-splines to test its utility (since their penalty is discrete) and then move on to the thin plate spline penalty.

In order to make this MDS approach competitive with \texttt{soap} (assuming that the above distortion adjustment is successful) a speed-up in the computation of the within-area distances must be achieved. At the moment this is the major bottleneck in the calculation since model fitting consists only of fitting a thin plate spline. 


\begin{table}[ht]
\centering
\begin{tabular}{c || c c c}
 & MDS & Soap film & Thin plate\\ 
\hline
Fit & 3.39225 & 10.87688 & 0.61492\\
Prediction & 4.80357 & 8.87535 & 0.10845\\
\end{tabular}
\label{ramsaytime}
\caption{Average time (in seconds) to fit a realisation of the modified Ramsay horseshoe for the three models considered above. Times are averaged over 100 realisations and were found using the elapsed time provided by \textsf{R}'s built-in \texttt{system.time} function.}
\end{table}


\begin{table}[ht]
\centering
\begin{tabular}{c || c c c c}
 & MDS & MDS (tensor) & Soap film & Thin plate\\ 
\hline
Fit & 84.85242 & 85.35168 & 36.24865 & 0.51721\\
Prediction & 155.4004 & 155.2419 & 44.22065 & 0.20994 \\
\end{tabular}
\label{wt2time}
\caption{Average time (in seconds) to fit a realisation of the peninsula domain for the four models considered above. Times are averaged over 100 realisations and were found using the elapsed time provided by \textsf{R}'s built-in \texttt{system.time} function.}
\end{table}


Speed-ups can be achieved by minimizing the number of calculations for within-area distance that are needed. In \cite{landmark} a method to calculate MDS coordinates using a sparse grid of landmark points is suggested; although the authors are sceptical about how this will perform with non-Euclidean distances. There is clearly some work here in finding some minimal set of landmark points for the within-area distance metric obtained from the above algorithm.

The calculation of within-area distances could be linked into a ``landmark''-type process by calculating the within-area distances between a set of points in a sparse grid, then saving those paths. The distances between desired points can then be found by modifying the path with the end points nearest to those of the desired points.


\bibliographystyle{plainnat}
\bibliography{mds-refs}

\end{document}
