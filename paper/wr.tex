\documentclass[10pt]{article}
%\documentclass[referee]{biom} 
%\usepackage{epsf}
%\usepackage{graphics}
%\usepackage{fullpage}
\usepackage{graphicx}
\usepackage{multirow}
\usepackage{makeidx}
\usepackage{latexsym}
\usepackage{bm}
\usepackage{amsmath}
\usepackage{amsthm}
\usepackage{amssymb}
\usepackage{setspace}
 
\onehalfspacing
%\doublespacing
\setlength{\textheight}{22.5cm}
\setlength{\textwidth}{6.47in}
\setlength{\oddsidemargin}{-1mm}
\setlength{\topmargin}{0.1cm}
\setlength{\evensidemargin}{-5mm}
\makeindex

\newcommand{\captionfonts}{\footnotesize}
\makeatletter  
\long\def\@makecaption#1#2{
  \vskip\abovecaptionskip
  \sbox\@tempboxa{{\captionfonts #1: #2}}
  \ifdim \wd\@tempboxa >\hsize
    {\captionfonts #1: #2\par}
  \else
    \hbox to\hsize{\hfil\box\@tempboxa\hfil}
  \fi
  \vskip\belowcaptionskip}
\makeatother


\newcommand {\hide}[1] {\typeout{ #1 }}
\newcommand{\beq}{\begin{equation}}
\newcommand{\eeq}{\end{equation}}
\newcommand{\dif}[2]{\frac{{\rm d} #1}{{\rm d} #2}}
\newcommand{\ildif}[2]{{\rm d} #1/{{\rm d} #2 }}
\newcommand{\ilpdif}[2]{\partial #1/{\partial #2 }}
\newcommand{\pdif}[2]{\frac{\partial #1}{\partial #2}}
\newcommand{\pddif}[3]{\frac{\partial^2 #1}{\partial #2 \partial #3}}
\newcommand{\ilpddif}[3]{\partial^2 #1/{\partial #2 \partial #3}}
\newcommand{\comb}[2]{\left (\begin{array}{c}{#1}\\{#2}\end{array}\right )}
\newcommand{\perm}[2]{^{#1}{\rm P}_{#2}}
\newcommand{\gfrac}[2]{\mbox{$ { \textstyle{ \frac{#1}{#2} }\displaystyle}$}}
\newcommand{\defn}{\begin{quote}{\bf Definition. }}
\newcommand{\edefn}{\end{quote}}
\newcommand{\thm}{\begin{theorem}}
\newcommand{\ethm}{\end{theorem}}
\newcommand{\R}{{\sf R}}
\newcommand{\s}{{\sf S}}
\newcommand{\fnzero}{\setcounter{footnote}{0}}
\newcommand{\bmat}[1]{\left ( \begin{array}{#1}}
\newcommand{\emat}{\end{array}\right )}
\newcommand{\E}{\mathbb{E}}
\newcommand{\ts}{^{\sf T}} 
\newcommand{\its}{^{\sf -T}}
\newcommand{\fv}{\hat{\bm{\mu}}}
\newcommand{\X}{{\bf X}}
\newcommand{\Xt}{\X\ts}
\newcommand{\y}{{\bf y}}
\newcommand{\A}{{\bf A}}
\newcommand{\Qf}{{\bf Q}_{\rm f}}
\newcommand{\bp}{{\bm \beta}}
\newcommand{\rsd}{\hat {\bm \epsilon}}
\newcommand{\grad}{\nabla_\beta}
\newcommand{\tr}[1]{{\rm tr}\left ( {#1} \right )}
\theoremstyle{definition}
\newtheorem*{defin}{Definition}
\theoremstyle{plain}
\newtheorem{theorem}{Theorem}
\newcommand{\rss}{{\cal S}}
\newcommand{\eps}[3]
{{\begin{center}
\rotatebox{#1}{\scalebox{#2}{\includegraphics{#3}}}
\end{center}}
}
\newcommand{\smsz}{\small}


\begin{document}


\title{within area distance paper...}

\author{David L. Miller and Simon N. Wood
%, \\ xxxxxxxxxxxxxxxxxxxxxxx \\ E-mail: 
}
%\footnote{Address for correspondence: }

\maketitle


%\begin{abstract}



%\vspace{0.4cm}
%\noindent \textbf{Keywords:} 

%\noindent \textbf{JEL Classification numbers:} C14, C25, D1.

%\end{abstract}

\section{TODO}
\begin{itemize}
\item write this
\item get relevant bits out of thesis
\end{itemize}

\section{Remember}
\begin{itemize}
\item W+R do the ridge penalty, why that's a good/bad idea.
\item artifact stuff, is that fair with the above taken into account?
\end{itemize}

\section{Finite area smoothing}





\section{Within area distance - why?}


\section{Our approach}


\section{Simulation experiment}

Simulation settings same as W+R paper. 200 realisations, 100 samples, noise=0.05, 40 knots selected using the \texttt{cover.design()} function in \texttt{fields}. Use a simplified boundary for speed: take every 5th vertex on the curved parts of the boundary. Predicated back over 

Compare the following models:

\begin{itemize}
	\item MDS+tprs
	\item MDS+tps
	\item soap film smoother
	\item WR+tps
\end{itemize}

Calculated MSE, also ``artefactiness'':
\begin{equation}
A= \int\int_\Omega  \Big( \frac{\partial^2 (f_m(x,y)-f_t(x,y))}{\partial x^2}\Big)^2 + \Big(\frac{\partial^2 (f_m(x,y)-f_t(x,y))}{\partial x \partial y}\Big)^2 + \Big(\frac{\partial^2 (f_m(x,y)-f_t(x,y))}{\partial y^2}\Big)^2 \text{d}x\text{d}y.
\end{equation}
where $f_m(x,y)$ is the model and  $f_t(x,y)$ is the truth. This is supposed to measure the wigglyness in the difference between the two, giving an idea of whether there are artefacts in the smooth.


\section{Conclusions}





\end{document}



