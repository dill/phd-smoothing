% mixture derivatives!
\label{app-mixderivs}

This appendix gives the derivations of the derivatives that were used to numerically maximize the likelihood for the mixture model detection functions in chapter \ref{chap-mmds}.

\section{Line transects}

Starting from the likelihood given in (\ref{lt-lik}), the derivatives with respect to the optimisation parameters are found.

\subsection{With respect to $\beta_{0j*}$}
For the intercept terms (also considering in the non-covariate case, these are just the parameters), the parameters have no effect outside of their mixture (ie. $\beta_{0j*}$ only has an influence on mixture component $j*$), so we can write:
\begin{equation*}
\frac{\partial l\left (\bm{\theta},\bm{\phi}; \mathbf{x},\mathbf{Z}\right )}{\partial \beta_{0j*}} = \sum_{i=1}^n \frac{1}{g\left (x_i,\mathbf{z}_i; \bm{\theta},\bm{\phi}\right )} \phi_{j*} \frac{\partial}{\partial \beta_{0j*}} g_{j*}\left (x_i,\mathbf{z}_i; \bm{\theta}_{j*}\right )  - \frac{\phi_{j*}}{\mu_i}  \frac{\partial}{\partial \beta_{0j*}} \mu_{ij*}.
\end{equation*}
Now, to first find\label{cor-e11} $\frac{\partial}{\partial \beta_{0j*}} g_{j*}\left (x_i,\mathbf{z}_i; \bm{\theta}_{j*}\right )$:
\begin{equation*}
\frac{\partial g_{j*}\left (x_i,\mathbf{z}_i; \bm{\theta}_{j*}\right )}{\partial \beta_{0j*}} = \frac{\partial}{\partial \beta_{0j*}} \exp\left ( -\frac{x_i^2}{2\sigma_{j*}^2} \right ),
\end{equation*}
applying the chain rule and remembering that $\sigma_{j*}$ is a (trivial) function of the $\beta_{0j}$s:
\begin{equation*}
\frac{\partial g_{j*}\left (x_i,\mathbf{z}_i; \bm{\theta}_{j*} \right )}{\partial \beta_{0j*}} = \left ( \frac{x_i}{\sigma_{j*}}\right )^2 \exp \left (-\frac{x_i^2}{2 \sigma_{j*}^2}\right ).
\end{equation*}

Expressing $\mu_{ij*}$ in terms of the error function:
\begin{align}
\frac{\partial \mu_{ij*}}{\partial \beta_{0j*}} &= \frac{\partial}{\partial \beta_{0j*}} \left \{ \sqrt{\frac{\pi}{2}} \sigma_{j*} \text{Erf}\left (\frac{w}{\sqrt{2\sigma_{j*}^2}}\right ) \right \} \notag, \\
&= \text{Erf}\left (\frac{w}{\sqrt{2\sigma_{j*}^2}}\right ) \frac{\partial}{\partial \beta_{0j*}} \left ( \sqrt{\frac{\pi}{2}} \sigma_{j*} \right ) + \sqrt{\frac{\pi}{2}} \sigma_{j*} \frac{\partial}{\partial \beta_{0j*}} \left \{ \text{Erf}\left (\frac{w}{\sqrt{2\sigma_{j*}^2}}\right ) \right \}. \label{app-mu-erf}
\end{align}
To find $\frac{\partial}{\partial \beta_{0j*}} \text{Erf}\left (\frac{w}{\sqrt{2\sigma_{j*}^2}}\right )$, note that we can write and then apply the chain rule:
\begin{align*}
\frac{\partial}{\partial \beta_{0j*}} \text{Erf}\left (\frac{w}{\sqrt{2\sigma_{j*}^2}}\right ) &= \frac{\partial}{\partial \beta_{0j*}} S \left \{ u(\sigma_{j*})\right \}, \\
&= \frac{\partial S(u)}{\partial u} \frac{\partial u(\sigma_{j*})}{\partial \sigma_{j*} } \frac{\partial \sigma_{j*}}{\partial \beta_{0j*}},
\end{align*}
where 
\begin{align*}
S(u) = \int_0^{u} \exp \left (-t^2 \right ) \text{d}t \quad \text{and} \quad u(\sigma_{j*})=\frac{w}{\sqrt{2\sigma_{j*}^2}}.
\end{align*}
Their derivatives being
\begin{align*}
\frac{\partial S(u)}{\partial u} = \frac{2}{\sqrt{\pi}} \exp(-u^2) \text{,} \quad \frac{\partial u(\sigma_{j*})}{\partial \sigma_{j*}} = -\frac{w}{\sqrt{2}}\sigma_{j*}^{-2}.
\end{align*}
Given these terms, it's just a case of multiplying them:
\begin{align*}
\frac{\partial S(u)}{\partial u} \frac{\partial u(\sigma_{j*})}{\partial \sigma_{j*} } \frac{\partial \sigma_{j*}}{\partial \beta_{0j*}} = - \sqrt{\frac{2}{\pi}} \frac{w}{\sigma_{j*}} \exp\left ( -\frac{w^2}{2\sigma_{j*}^2} \right ).
\end{align*}
Substituting into (\ref{app-mu-erf}):
\begin{equation*}
\frac{\partial \mu_{ij*}}{\partial \beta_{0j*}} =  \mu_{ij*} - w \exp\left ( -\frac{w^2}{2\sigma_{j*}^2} \right ).
\end{equation*}
Finally, the derivative is:
\begin{equation*}
\frac{\partial l(\bm{\theta}, \bm{\phi}; \mathbf{x},\mathbf{Z})}{\partial \beta_{0j*}} = \sum_{i=1}^n \left ( \frac{x_i}{\sigma_{j*}}\right )^2 \phi_{j*} \frac{g_{j*}(x_i,\mathbf{z}_i; \bm{\theta}_{j*})}{g(x_i,\mathbf{z}_i; \bm{\theta},\bm{\phi})}  - \frac{\phi_{j*}}{\mu_i} \left \{ \mu_{ij*} - w g_{j*}(w,\mathbf{z}_i; \bm{\theta}_{j*}) \right \}.
\end{equation*}



\subsection{With respect to $\beta_{k*}$}

Derivatives with respect to the common covariate parameters are found in a similar way to above. The expressions are slightly more complicated since the $\beta_k$s affect all of the mixture components.
\begin{equation*}
\frac{\partial l(\bm{\theta},\bm{\phi}; \mathbf{x},\mathbf{Z})}{\partial \beta_{k*}} = \sum_{i=1}^n \left \{ \frac{1}{g(x_i,\mathbf{z}_i; \bm{\theta},\bm{\phi})} \sum_{j=1}^J \phi_j \frac{\partial}{\partial \beta_{k*}} g_j(x_i,\mathbf{z}_i; \bm{\theta}_j) - \frac{1}{\mu_i} \sum_{j=1}^J \phi_j \frac{\partial}{\partial \beta_{k*}}\mu_{ij}\right \}.
\end{equation*}
Every $\sigma_{j}$ is a function of the $\beta_{k}$s, so:
\begin{align*}
\frac{\partial \sigma_{j}}{\partial \beta_{k*}} &= \frac{\partial}{\partial \beta_{0j*}} \exp \left ( \beta_{0j} + \sum_{k=1}^K z_{ik} \beta_{k}\right ),\\
&= z_{ik*} \exp \left ( \beta_{0j} + \sum_{k=1}^K z_{ik} \beta_{k}\right ),\\
&= z_{ik*}\sigma_{j}.
\end{align*}
Hence:
\begin{equation*}
 \frac{\partial}{\partial \beta_{k*}} \exp\left ( -\frac{x_i^2}{2\sigma_{j}^2} \right ) = z_{k*} \left ( \frac{x_i}{\sigma_{j}}\right )^2 \exp \left (-\frac{x_i^2}{2 \sigma_{j}^2}\right ) = z_{k*} \left ( \frac{x_i}{\sigma_{j}}\right )^2 g_j(x_i,\mathbf{z}_i; \bm{\theta}_j).
 \label{detfct-deriv-k}
\end{equation*}
And so for the $\mu_{ij}$s:
\begin{equation*}
\frac{\partial \mu_{ij}}{\partial \beta_{k*}} = z_{ik*} \left \{ \mu_{ij} - w \exp\left ( -\frac{w^2}{2\sigma_{j}^2} \right ) \right \}.
\end{equation*}
The derivative is then:
\begin{align*}
\frac{\partial l(\bm{\theta},\bm{\phi}; \mathbf{x},\mathbf{Z})}{\partial \beta_{k*}} = \sum_{i=1}^n & \left [ \frac{1}{g(x_i,\mathbf{z}_i; \bm{\theta},\bm{\phi})} \sum_{j=1}^J \phi_j  z_{k*} \left ( \frac{x_i}{\sigma_{j}}\right )^2 g_j(x_i,\mathbf{z}_i; \bm{\theta}_j) \right. \\
& - \left. \frac{1}{\mu_i} \sum_{j=1}^J \phi_j z_{ik*} \left \{ \mu_{ij} - w g_j(x_i,\mathbf{z}_i; \bm{\theta}_j) \right \} \right ].
\end{align*}

\subsection{With respect to $\alpha_{j*}$}

First note that we can write the likelihood (\ref{lt-lik}) as:
\begin{align*}
l(\bm{\theta},\bm{\phi}; \mathbf{x},\mathbf{Z}) = \sum_{i=1}^n  & \left [ \log \left \{ \sum_{j=1}^{J-1} \phi_j g_j(x_i,\mathbf{z}_i; \bm{\theta}_j) + \left (1-\sum_{j=1}^{J-1} \phi_j \right ) g_J(x_i,\mathbf{z}_i; \bm{\theta}_J) \right \} \right. \\
& \left. - \log \left \{ \sum_{j=1}^{J-1} \phi_j \mu_{ij} + \left (1-\sum_{j=1}^{J-1} \phi_j \right ) \mu_{ij}  \right \} \right ].
\end{align*}
The derivatives with respect to the $\alpha_{j*}$ of this expression are then:
\begin{align}
\frac{\partial l(\bm{\theta},\bm{\phi}; \mathbf{x},\mathbf{Z})}{\partial \alpha_{j*}} = & \left [ \sum_{i=1}^n \frac{1}{g(x_i,\mathbf{z}_i; \bm{\theta},\bm{\phi})} \left \{ \sum_{j=1}^{J-1} g_j(x_i,\mathbf{z}_i; \bm{\theta}_j) \frac{\partial \phi_j}{\partial \alpha_{j*}}  -g_J(x_i,\mathbf{z}_i; \bm{\theta}_J) \sum_{j=1}^{J-1}  \frac{\partial \phi_j}{\partial \alpha_{j*}} \right \} \right. \notag \\
&- \left. \frac{1}{\mu_i} \left (\sum_{j=1}^{J-1} \mu_{ij} \frac{\partial \phi_j}{\partial \alpha_{j*}} - \mu_{iJ} \sum_{j=1}^{J-1}   \frac{\partial \phi_j}{\partial \alpha_{j*}} \right ) \right ]. \label{app-lik-alphad}
\end{align}
Finding the derivatives is then simply a matter of finding the derivatives of $\phi_{j}$ with respect to $\alpha_{j*}$ and substituting them back into (\ref{app-lik-alphad}).
\begin{equation*}
\frac{\partial \phi_j}{\partial \alpha_{j*}} = \frac{\partial}{\partial \alpha_{j*}}F \left (\sum_{p=1}^j e^{\alpha_p} \right ) - \frac{\partial}{\partial \alpha_{j*}} F \left (\sum_{p=1}^{j-1} e^{\alpha_p} \right ).
\end{equation*}
Looking at each of the terms:
\begin{equation*}
\frac{\partial}{\partial \alpha_{j*}} F \left (\sum_{p=1}^j e^{\alpha_p} \right )=A_{j}=\begin{cases}
e^{\alpha_{j*}}f \left (\sum_{p=1}^j e^{\alpha_p} \right )& \text{for $j\geq j*$},\\
0 & \text{for $j<j*$}.
\end{cases}
\end{equation*}
and
\begin{equation*}
\frac{\partial}{\partial \alpha_{j*}} F \left (\sum_{p=1}^{j-1} e^{\alpha_p} \right )=A_{(j-1)}=\begin{cases}
e^{\alpha_{j*}}f \left (\sum_{p=1}^{j-1} e^{\alpha_p} \right )& \text{for $j-1\geq j*$},\\
0 & \text{for $j-1<j*$}.
\end{cases}
\end{equation*}
So
\begin{equation*}
\frac{\partial \phi_j}{\partial \alpha_{j*}} = A_j - A_{j-1}.
\end{equation*}
Substituting these back into (\ref{app-lik-alphad}) and re-arranging gives:
\begin{align*}
\frac{\partial l(\bm{\theta},\bm{\phi}; \mathbf{x},\mathbf{Z})}{\partial \alpha_{j*}} = \sum_{i=1}^n & \left [ \frac{1}{g(x_i,\mathbf{z}_i; \bm{\theta},\bm{\phi})} \sum_{j=1}^{J-1} \left (A_j - A_{j-1} \right ) \left \{ g_j(x,\mathbf{z}_i; \bm{\theta}_j) - g_J(x,\mathbf{z}_i; \bm{\theta}_J) \right \} \right. \\
&- \left. \frac{1}{\mu_i} \sum_{j=1}^{J-1} \left (A_j - A_{j-1}\right ) \left (\mu_{ij} - \mu_{iJ} \right ) \right ].
\end{align*}

\section{Point transects}

\subsection{With respect to $\beta_{0j}$}

Starting with the likelihood in (\ref{pt-lik}), one can see that we obtain:
\begin{align*}
\frac{\partial l(\bm{\theta}, \bm{\phi}; \mathbf{r},\mathbf{Z})}{\partial \beta_{0j*}}  &= \sum_{i=1}^n \left \{ \frac{\partial}{\partial \beta_{0j*}} \log \sum_{j=1}^J \phi_j g_j(r_i,\mathbf{z}_i; \bm{\theta}_j) - \frac{\partial}{\partial \beta_{0j*}}\log \sum_{j=1}^J \phi_j \nu_{ij}\right \},\\
&= \sum_{i=1}^n \left \{ \frac{ \phi_{j*} \frac{\partial}{\partial \beta_{0j*}}  g_{j*} (r_i,\mathbf{z}_i; \bm{\theta}_j)}{g(r_i,\mathbf{z}_i; \bm{\theta}, \bm{\phi})} - \frac{ \phi_{j*}\frac{\partial}{\partial \beta_{0j*}}  \nu_{ij*} }{ \sum_{j=1}^J \phi_j \nu_{ij}}\right \},
\end{align*}
the first part of which (the derivatives of the detection function) are as in the line transect case. The derivatives of $\nu_{ij}$ are simpler in the point transect case, since there is an easy analytic expression for $\nu_{ij}$ when $g_j$ is half-normal :
\begin{equation*}
\nu_{ij} = 2 \pi \sigma_{ij}^2 \left \{ 1-\exp (-w^2/2\sigma_{ij}^2 ) \right \},
\end{equation*}
then simply applying the product rule yields:
\begin{equation*}
\frac{\partial \nu_{ij}}{\partial \beta_{0j*}} = 2 \left \{ \nu_{ij*} + \pi w^2 g_{j*}(w) \right \}.
\end{equation*}
Substituting this into the above expression:
\begin{equation*}
\frac{\partial l(\bm{\theta}, \bm{\phi}; \mathbf{r},\mathbf{Z})}{\partial \beta_{0j*}}  = \sum_{i=1}^n \left [ \frac{ \phi_{j*} (r_i/\sigma_{j*})^2 g_{j*}(r_i,\mathbf{z}_i; \bm{\theta}_{j*})}{g(r_i,\mathbf{z}_i; \bm{\theta}, \bm{\phi})} - \frac{ \phi_{j*} 2 \left \{ \nu_{j*} + \pi w g_{j*}(w) \right \} }{ \sum_{j=1}^J \phi_j \nu_{ij}}\right ].
\end{equation*}

\subsection{With respect to $\beta_{k*}$}

Again working from (\ref{pt-lik}), we obtain:
\begin{align*}
\frac{\partial l(\bm{\theta}, \bm{\phi}; \mathbf{r},\mathbf{Z})}{\partial \beta_{k*}}  &= \sum_{i=1}^n \left \{ \frac{\partial}{\partial \beta_{k*}} \log \sum_{j=1}^J \phi_j g_j(r_i,\mathbf{z}_i; \bm{\theta}_j) - \frac{\partial}{\partial \beta_{k*}}\log \sum_{j=1}^J \phi_j \nu_{ij}\right \}\\
&= \sum_{i=1}^n \left \{ \frac{ \sum_{j=1}^J \phi_{j} \frac{\partial}{\partial \beta_{k*}}  g_{j} (r_i,\mathbf{z}_i; \bm{\theta}_j)}{g(r_i,\mathbf{z}_i; \bm{\theta}, \bm{\phi})} - \frac{ \sum_{j=1}^J \phi_{j}\frac{\partial}{\partial \beta_{k*}}  \nu_{ij} }{ \sum_{j=1}^J \phi_j \nu_{ij}}\right \}.
\end{align*}
The derivatives of $g_j$ are as in (\ref{detfct-deriv-k}). For $\nu_{ij}$:
\begin{equation*}
\frac{\partial \nu_{ij}}{\partial \beta_{k*}} =  2z_{ik*} \left \{ \nu_{ij} - \pi w^2 g_j(w) \right \}.
\end{equation*}
Putting that together:
\begin{equation*}
\frac{\partial l(\bm{\theta}, \bm{\phi}; \mathbf{r},\mathbf{Z})}{\partial \beta_{k*}}  = \sum_{i=1}^n \left [ \frac{ \sum_{j=1}^J \phi_{j} z_{k*} \left ( \frac{x_i}{\sigma_{j}}\right )^2 g_j(x_i,\mathbf{z}_i; \bm{\theta}_j)}{g(r_i,\mathbf{z}_i; \bm{\theta}, \bm{\phi})} - \frac{ \sum_{j=1}^J \phi_{j}2z_{ik*} \left \{ \nu_{ij} - \pi w^2 g_j(w) \right \} }{ \sum_{j=1}^J \phi_j \nu_{ij}}\right ].
\end{equation*}
