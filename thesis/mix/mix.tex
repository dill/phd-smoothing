\section{Introduction}

Repeat what was in the intro a bit

Why do this?

%%%%%%%%%%%%%%%%%%%%%%%%%%%%%%%%%%%%%%%%%%%%%%%%%%%
\section{General formulation}

The general principle here is to replace the ``key function plus adjustment terms'' model for the detection function with a mixture model. The simplest example would be to define $g$ as some finite weighted sum of half-Normal distributions:
\begin{equation*}
g(x;\bm{\sigma},\bm{\pi}) = \sum_{j=1}^J \pi_j \exp \Big( \frac{-x^2}{2 \sigma_j^2}\Big).
\end{equation*}
Where the mixture proportions, $\pi_j$, have the property $\sum_{\forall j}\pi_j=1$ ($\bigcup_{\forall j} \pi_j=\bm{\pi}$). The (scale) parameters $\bm{\sigma}=(\sigma_1,\sigma_2,\dots,\sigma_J)$


\subsection{Line transects}

\subsection{Point transects}

\subsection{Covariate models}


\subsection{Detection probability}

$P_a$, per observation $p$, Var and CV.

\subsection{GoF}

%%%%%%%%%%%%%%%%%%%%%%%%%%%%%%%%%%%%%%%%%%%%%%%%%%%
\section{Implementation}

\subsection{Parametrisation}

\subsection{Fitting}

\subsection{Starting values}

\subsection{Derivatives}



%%%%%%%%%%%%%%%%%%%%%%%%%%%%%%%%%%%%%%%%%%%%%%%%%%%
\section{Testing}

\subsection{Simulations}

\subsection{Real data}


%%%%%%%%%%%%%%%%%%%%%%%%%%%%%%%%%%%%%%%%%%%%%%%%%%%
\section{Conclusion}