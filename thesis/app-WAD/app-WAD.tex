%%
\label{chap-WAD}


Here some properties of the algorithm given in \secref{mdsdist} are given.

\section{Preliminaries}

To begin with some preliminaries. Notation is as in \secref{mdsdist}, but is refreshed here. In this section a proof of the uniqueness of the shortest path in a simple polygon is given.

\subsection{Notation}

DEFINE everything here: COMBINE!

$p_1$ and $p_2$ always refer to the two points that we want to find the shortest path between

$\Gamma$ is the polygon that we are interested in

$v_i$ refers to a point in a path within the domain, it may or may not be a vertex of $\Gamma$.

$\mathcal{P}_1$ and $\mathcal{P}_2$ are the initial paths between $p_1$ and $p_2$ that trace the boundary.

$\mathcal{P}$ is the working path between $p_1$ and $p_2$.


$\mathcal{S}_{i}$ is a section of a path.

\subsection{Terminology}

The word triangle here is used to refer to the case in which three points, $(v_i,v_{i+1},v_{i+2})$ are vertices of a triangle all the adges of which lie within $\Gamma$. These are the cases in which the DELETE would simplify $(v_i,v_{i+1},v_{i+2})$ to $(v_i,v_{i+2})$.


\subsection{Uniqueness of shortest path}
\label{app-unique-sp}
Propostition: there is one, unique shortest path between point $p_1$ and $p_2$ within a simple polygon (ie. the polygon has no holes), $\Gamma$.

This is simple to see since if the shortest path were non-unique then there would exist $\mathcal{P}_A$ and  $\mathcal{P}_B$ which were both shortest paths between $p_1$ and $p_2$. It would then be the case there there would be two points $v_1^*$ and $v_2^*$ say where the paths started to differ and ended differing (these could be $p_1$ and $p_2$): the points at which $\mathcal{P}_A$ and  $\mathcal{P}_B$ become disjoint. Call the two paths that lie between $v_1^*$ and $v_2^*$ $\mathcal{S}_{AB}$ and $\mathcal{S}_{BA}$. 

Joining up $\mathcal{S}_{AB}$ and $\mathcal{S}_{BA}$ at $v_1^*$ and $v_2^*$ forms a loop, $\mathcal{L}$, say. There is nothing in the middle of $\mathcal{L}$ (since $\Gamma$ has no holes), in which case there must be a kink in at least one of the paths (since they are not identical, at least one of them must not be a straight line between $v_1^*$ and $v_2^*$), this means that there is a triangle in (at least) one of the paths, which can be straightened ($\mathcal{L}$ does not contain any obstacle stopping this) so the path is not a shortest path and can be shortened. This is a contradiction, therefore the shortest path must be unique. Figure \ref{app-WAD-unique-dia} may be helpful.

This proof was adapted from the one given in lecture notes by Leonidas Guibas available at \url{http://graphics.stanford.edu/courses/cs268-09-winter/}.


\begin{figure}
\centering
\includegraphics[width=3in]{app-WAD/figs/unique-path-dia.pdf} \\
\caption{Illustration of some of the terms used in the proof that there is always a unique shortest path. $\mathcal{P}_A$ and  $\mathcal{P}_B$ are given by the red and blue lines. The loop ($\mathcal{L}$) is given by the dotted lines; the red part is  $\mathcal{S}_{AB}$ and the blue part  $\mathcal{S}_{BA}$.}
\label{app-WAD-unique-dia}
% generated by thesis/app-WAD/figs/unique-shortest-path.R
\end{figure}



\section{Properties}

\subsection{The algorithm terminates}

The condition to be fulfilled for the algorithm to terminate is that in two consecutive runs the path does not change.

First note that the ALTER step can act as (a less efficient) DELETE step: $\mathcal{P}_{I}$ would be the triplet $(v_i,v_{i+1},v_{i+2})$ (a triangle) then the DELETE substep removes $v_{i+1}$, giving $(v_i,v_{i+2})$), which would then be $\mathcal{P}_{ID}$ which could be inserted into the path. For this reason we only need to consider the ALTER step here.

For the path not to change we just require that both DELETE and ALTER steps do not change the path. In the best case this happens when the path is the shortest, since there are no modifications we can make to obtain a shorter path (since the path is unique, once it has been reached, there are no other possible modifications that can be made).

In the worst case one might imagine that the algorithm could get caught in a loop. This cannot happen since this would require that the ALTER step proposed shorter paths at every iteration (since otherwise they would not be accepted) \textit{ad infinitum}. This would imply that at every stage a shorter path were possible (since only shorter, not equal length modifications are accepted), clearly this cannot be the case.

\subsection{The algorithm terminates at the shortest path}


As mentioned above, the path must get shorter at each iteration. 

The algorithm terminates.

It can only terminate when the path cannot get any shorter.

To show that the overall path is the shortest, can just show that each individual 3-ple is optimal.


If the path is shortest then each sub-path (triplet) must be optimal too (otherwise we could run ALTER or DELETE on it and get a shorter path.)

So it satistys o show that each triplet is optimal.


Given $(v_{i}, v_{i+1}, v_{i+2})$ there are four possible options
\begin{enumerate}
   \item Trivially $v_{i}=v_{i+1}=v_{i+2}$, in which case the DELETE step will remove all but $v_{i}$.
   \item One of $v_{i}=v_{i+1}$, $v_{i}=v_{i+2}$ or $v_{i+1}=v_{i+2}$, again the DELETE step will remove the duplicate vertex.
   \item $(v_{i}, v_{i+1}, v_{i+2})$ form a triangle. The DELETE step will remove $v_{i+1}$.
   \item $(v_{i}, v_{i+1}, v_{i+2})$ do not form a triangle. In this case there is some ``obstacle'' between $v_{i}$ and $v_{i+2}$. In this case ALTER is used to navigate the path around the obstacle.
\end{enumerate}

In all but the last case it is clear by definition that the new path will be shorter. The final case is now covered in detail; in essence this amounts to showing that subpaths proposed by ALTER are locally optimal.

All cases where ALTER can propose new (non-trivial, in the DELETE sense) paths are more complicated versions of the situation shown in the first panel of figure \ref{app-WAD-alterstep-dia}. 


The key features here being that the point $v_{i+1}$ can be replaced with the point $v_{i+1}^*$. This is the simplest case when the ALTER step can be used. 

However all situations reduce to an essentially similar situation. The grey area can be any shape and 


TKTKTK define obstacle here

These situations boil down to two options
\begin{enumerate}
   \item The grey obstacle is a concave, ie. the shortest path is simply a staight line between two points on the boundary which avoid all of the other vertices on the boundary that make up the obstacle (the second panel in figure \ref{app-WAD-alterstep-dia}. In this case since we can split the area ``under'' the two end points into triangles, the DELETE step makes short work of the extra path elements.
   \item The grey obstacle is convex after removing any triangles in that bounadry segment, ie. more than the two end points are required to form a path around the obstacle. This is shown in the third panel in figure \ref{app-WAD-alterstep-dia}. In this case, tracing the boundary and removing triangles in the path will still yield the shortest path. We can always obtain the shortest path by tracing the boundary and removing traingles since there cannot be any ALTER steps needed (since that would imply holes) and then the path can simply be minimized by DELETE steps.
\end{enumerate}


\begin{figure}
\centering
\includegraphics[width=6in]{app-WAD/figs/alterstep-proof.pdf} \\
\caption{three panels!}
\label{app-WAD-alterstep-dia}
% generated by thesis/app-WAD/figs/makealterstepdia.R
\end{figure}


Since the ALTER step is optimal locally that shows that all triplets are are optimal and ehnce the whole path is optimal. So paths are shortest paths.



\subsection{The computational complexity of the algorithm}





