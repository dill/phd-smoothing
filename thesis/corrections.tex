[[THERE ARE TKTKTKs to look at at the end!!!]]

\textbf{Bold} indicates a query.

Page numbers are updated ones.

Where multiple contiguous pages have been changed, only the first is referenced below.


\section{From e-mail 30$^\text{th}$ January 2012.}

The points raised in the e-mail have been addressed as follows:
\begin{itemize}
\item Corrections to ``minor mathematical and grammatical errors throughout the thesis'' according to the two lists sent are itemized in sections \ref{richard} and \ref{shaw}, below.
\item Restructured thin plate (\pref{cor-tprs}) and cubic spline (\pref{cor-cubic}) explanation.
\item Elaboration on soap film and added stronger links to chapter 2.
\textit{Chapter 1 in general needs to be fleshed out, so as better to set the stage for the remainder of the thesis. For example, the soap film smoother is mentioned almost parenthetically at present, whereas it is adopted as a benchmark for the remainder of Part 1 of the thesis. We do not necessarily ask that all of the details are moved from Chapter 2 (where they appear at present) to Chapter 1: however, some intuition needs to be given, and the key role of this technique needs to be made clear. This would also provide the opportunity to motivate the inclusion of Chapter 2 much more convincingly: the chapter can be presented as a detailed introduction to, and illustration of, the use of the soap film smoother which then sets the stage for the subsequent development.}
\item 
\begin{itemize}
	\item \pref{cor-r22} explained that \texttt{soap} and \texttt{mgcv} were used.
	\item \pref{cor-soft1} clarified that all the models were fitted using existing libraries.
	\item \pref{sc-sims} explains that the MATLAB package \textit{SC Toolbox} is used for the \sch\ mapping and \texttt{mgcv} and \texttt{soap} are used for smoothing.
\end{itemize}
\textit{Throughout the thesis, we would like you to be clear about what software is being used: at present, it is not always clear whether you are picking up and running with �off-the shelf� software (which turned out to be the case for the Schwarz-Christoffel transform) or whether you had to write new code yourself. The work of others should always be duly acknowledged; and you shouldn?t be shy about advertising your own contributions! }
\item \textit{In Chapter 4, the discussion of the �adjustment� {\cal L} needs to be clarified to make clear that it is heuristic: at present, it is presented as somehow deeper than this. As discussed during the oral examination, one possibility would be to start by considering the idea of using the Jacobian of the transformation, explaining why (as indicated by the candidate during the oral) this didn?t work, and then moving on to the �engineering� solution.}
\item \textit{Chapter 7 needs to be restructured: at present the summary appears in Section 7.8, and is followed by two additional sections!}
\end{itemize}


\section{Minor corrections from Richard Chandler}
\label{richard}

\begin{enumerate}
\item p. 26 Changed ``this data'' to ``these data''
\item p. 26 Changed to talk about trend surface, rather than process.
\item p. 28 ``Complementary''
\item p. 29, 30, [[TKTKTK p66:13, p227:6 find at end!]] ``effect'' $\rightarrow$ ``affect''.
\item p. 30, changed caption to say that the function begins to interpolate the data.
\item p. 30, difference between thin plate splines and thin plate regression splines given from the outset.
\item p. 38 ``datum'' $\rightarrow$ ``data''
\item p. 40. now reads as requested.
\item p. 41 ``PIRLS'' $\rightarrow$ ``IRLS'', included definition of $\mathcal{V}_g$ in the GAM case.
\item Section 1.3 title ``practice'' $\rightarrow$ ``practise'', same p. 183
\item p. 42 corrected two typos
\item (1.10) corrected Brier definition to have the data in.
\item p. 44 explained EDF--complexity link better.
\item p. 45 corrected two typos. \textbf{qualification for the ``flawed'' nature of existing models is in the following text both in section 1.4.1 and the abstract}
\item p. 49 corrected typo
\item p. 50 unsure what I meant by ``leakage as a breakdown of stationarity'', I've elaborated further on how these methods work as an introduction to the MDS stuff that comes later on.
\item p. 50 corrected typo
\item \textit{Chapter 2} \pref{cor-r18} corrected typo.
\item \pref{cor-r19} Added comment about missing data, changed plot to highlight those cells with no data to be blue.
\item \pref{cor-r20} Expanded section on Tweedie distribution.
\item \pref{cor-r21} Explained why lat/long lead to anisotropy.
\item \pref{cor-r22} Said that \texttt{soap} and \texttt{mgcv} were used to construct the model.
\item \pref{cor-r23} Added an extra line explaining that $\lambda_\text{int}$ and $\lambda_\text{bnd}$ are actually estimated rather than $\lambda_\text{space}$. I'm inclined to leave (2.3) as having only $\lambda_\text{space}$ in as it's true in general for tensor products and it makes the explanation of the tensor product simpler. What do you think?
\item \pref{cor-r24} Added assumptions for posterior.
\item \pref{cor-r25} Added Richard's paper to the bibliography.
\item \pref{cor-r26} Added figure showing spatial groups and changed caption accordingly. Caption also claimed that the grid was 10x10, it was 5x5. Added comment on the line in the scale-location plot, which does indeed correspond to zeros in the data, coloured these data grey.
\item \pref{cor-r27} Added comment about how tight the intervals in figure 2-5 are.
\item \pref{cor-r28} Added timing for model fitting.
\item \textit{Chapter 3} \pref{cor-r29}
\begin{itemize}
	\item \pref{cor-r29-1} Tidied up and reordered these paragraphs so that we know we're talking about complex numbers before we start.
	\item \pref{cor-r29-2}
	\item \pref{cor-r29-3} ``can be computed''.
	\item \pref{cor-r29-4}
	\item \pref{cor-r29-5}
	\item \pref{cor-r29-6} Removed sentence, as advised by SS, below.%corrected to ``known''.
\end{itemize}
\item \textit{Chapter 4} \pref{cor-r30}, \pref{cor-r30-1} ``principle'' $\rightarrow$ ``principal''.
\item \pref{cor-r31}
\begin{itemize}
	\item \pref{cor-r31-1}
	\item \pref{cor-r31-2}
	\item \pref{cor-r31-3}
	\item \pref{cor-r31-4}
	\item \pref{cor-r31-5}
	\item \pref{cor-r31-6}
	\item \pref{cor-r31-7}
\end{itemize}
\item \pref{cor-r32} ``is show'' $\rightarrow$ ``is shown''.
\item \pref{cor-r33} ``novel'' $\rightarrow$ ``a novel''.
\item \pref{cor-r34}  ``the'' $\rightarrow$ ``that''.
\item \pref{cor-r35} 
\item \pref{cor-r36} Added explanation of the change in surface from the earlier peninsula domain.
\item \pref{cor-r37} ``sever'' $\rightarrow$ ``severe''
\item \pref{cor-r38} Changed axis labels to be correct.
\item \pref{cor-r39}
\item \pref{cor-r40}
\item \pref{cor-r41} Removed $\dagger$, relic of previous document.
\item \pref{cor-r42} 
\item \pref{cor-r43} Referenced back to section 1.1.
\item \pref{cor-r44} response is assumed Gamma distributed.
\item \pref{cor-r45} Corrected grammar.
\item \pref{cor-r46} Bug in plotting code caused the dimension to be shifted, corrected.
\item \pref{cor-r47} ``an as such'' $\rightarrow$ ``and as such''.
\item \pref{cor-r48} \textit{Chapter 5} ``massive'' $\rightarrow$ ``a massive''.
\item \pref{cor-r49} 
\item \pref{cor-r50} ``fequencies'' $\rightarrow$ ``frequencies''.
\item \pref{cor-r51} 
\item \pref{cor-r52} Boxplots are in groups of 5!
\item \pref{cor-r53} Made this sentence make more sense.
\item \pref{cor-r54}
\item \pref{cor-r55}
\item \pref{cor-r56} \textit{Chapter 6} 
\end{enumerate}

``'' $\rightarrow$ ``''.

\section{Minor corrections from Simon Shaw}
\label{shaw}

\subsection{Chapter 1}
\begin{itemize}
\item p. 29 Added $\dots$ (twice).
\item p. 29 Reference corrected.
\item p. 30 Added $\dots$, removed emboldening.
\item p. 32 Changed ``Say we put'' to jump straight into talking about the matrix.
\item p. 34 ``practise'' $\rightarrow$``practice'', throughout.
\item p. 34,35 B-spline definition now uses $x$ throughout. Summation corrected and $i$ terms removed.
\item p. 35 Objective function corrected.
\item p. 36 Removed self-reference.
\item p. 36 Corrected definition of cubic splines.
\item \textbf{p. 36 ``p36 $\beta_j$ and $\delta_j$ should be defined in relation to the form of $f(x)$''. Can you elaborate on what this means?}
\item p. 36 [[TKTKTK 37?]]  Short discussion of cyclic splines added.
\item p. 36 Changed definition to $\delta_j = \frac{\partial^2 f(x)}{\partial x^2}\vert_{x=x_j^*}$
\item p. 37 Tensor product definition now has two different size bases.
\item p. 38,39 defined $\mathbf{\hat{f}}$.
\item p. 39 ``leasst'' $\rightarrow$ ``least''.
\item p. 40 $\bm{\beta}^{[k]} \rightarrow \bm{\hat{\beta}}^{[k]}$
\item p. 40 added $\dots$
\item p. 41 smoothing parameter does need to be a vector, changed later references
\item p. 41 corrected reference
\item p. 42 added brackets to references
\item p. 45 cleared up explanation of a complex boundary with an example
\item p. 45 fixed typo
\item p. 47 fixed typo
\item p. 48 removed
\item p. 49 fixed typo
\end{itemize}

\subsection{Chapter 2}
\begin{itemize}
\item \pref{cor-2s1} Corrected.
\item \pref{cor-2s2} Removed.
\item \pref{cor-2s3} \Secref{mds-faster} deals with ensuring that the model we're using is fast enough not to cause frustration to users. Should I reference this here?
\end{itemize}

\subsection{Chapter 3}
\begin{itemize}
\item \pref{cor-3s1-1} Changed $\Gamma$ to $W$. \pref{cor-3s1-3} stated that $W$ is the inside of $\Gamma$.
\item \pref{cor-3s2} Removed references to $\varphi$ at this point.
\item \pref{cor-3s3} Expanded opening paragraph(s) to discuss the Ramsay horseshoe in a bit more depth.
\item \pref{cor-3s4} Replaced figure 3-1 with 3-5; corrected $x$ to $w$ and $w^*$ in caption; reversed diagram to be same ordering as others, later. \label{cor-3s4} Added some elaboration here.
\item \pref{cor-3s5} Cleared up caption and figure. Corrected (and added) line label. Caption: $w_6^* \rightarrow w_6$. 
\item \pref{cor-3s6} Corrected.
\item \pref{cor-3s7} Cleared up caption and figure. Changed line label. ``upper half-plane'' $\rightarrow$ ``unit disc''
\item \pref{cor-3s8-1} It should be a polygon, that shouldn't have implied that wasn't the case -- corrected. \pref{cor-3s8-2} Clarified $\varphi$ and $\varphi^{-1}$ are found. \pref{cor-3s8-3} $\varphi \rightarrow \varphi^{-1}$.
\item \pref{cor-3s9} Corrected angles in diagram. Increased shading.
\item \pref{cor-3s10} Corrected as suggested.
\item \pref{cor-3s11} Final $\varphi$ should be $\varphi^{-1}$.
\item \pref{cor-3s12} Corrected ``may'' $\rightarrow $ ``many''.
\item \pref{cor-3s13} 
\item \pref{cor-3s1-1} Changed $\Gamma$ to $W$. \pref{cor-3s14-1} First sentence the right way around. \pref{cor-3s14-2} $w_0 \rightarrow w^*_0$.  \pref{cor-3s14-3} Talking about calculation numerically, put into separate paragraph.
\item \pref{cor-3s15} Changed $\Gamma$ to $W$.
\item \pref{cor-3s16} Put this computational bit in a separate paragraph, cleared up who solved the numerical issue.
\item \pref{cor-3s17} 
\item \pref{cor-3s18-1} Removed ref to (3.3), twice. \pref{cor-3s18-2} I've reordered this paragraph, putting the rectangle first and referencing where that comes from. I've left the unit disc but in since I do show examples of the unit disc later. 
\item \pref{cor-3s19} Removed sentence, don't really need to know this.
\item \pref{cor-3s20} 
\end{itemize}

\subsection{Chapter 4}
\begin{itemize}
\item \pref{cor-4s1}
\item \pref{cor-4s2} Emboldened.
\item \pref{cor-4s3} Added ``the''.
\item \pref{cor-4s4} Re-ordered figures.
\item \pref{cor-4s5} Corrected caption.
\item \pref{cor-4s6} Cleared up contradiction.
\item \pref{cor-4s7} Corrected value in table.
\item \pref{cor-4s8} Grammar correction.
\item \pref{cor-4s9} Changed equation as suggested.
\item \pref{cor-4s10} Corrected line break.
\item \pref{cor-4s11}
\end{itemize}

\subsection{Chapter 5}
\begin{itemize}
\item 
\end{itemize}

\subsection{Chapter 6}
\begin{itemize}
\item 
\end{itemize}

\subsection{Chapter 7}
\begin{itemize}
\item 
\end{itemize}

\subsection{Chapter 8}
\begin{itemize}
\item 
\end{itemize}

\section{Extras}

\begin{itemize}
\item p. 36, added limits to the penalty for cubic splines.
\item p. 26 " (in the sense that they take the value one at one knots and zero at all others)" $\rightarrow$  (in the sense that they take the value one at one knot and zero at all others)
\item p. 36 Noted that the $b_j$s that result are implicit.
\item Chapter 3 ``disk'' $\rightarrow$ ``disc''.
\end{itemize}

