[[THERE ARE TKTKTKs to look at at the end!!!]]

\textbf{Bold} indicates a query.

Page numbers are updated ones.

Where multiple contiguous pages have been changed, only the first is referenced below.


\section{From e-mail 30$^\text{th}$ January 2012.}

The points raised in the e-mail have been addressed as follows:
\begin{itemize}
\item Corrections to ``minor mathematical and grammatical errors throughout the thesis'' according to the two lists sent are itemized in sections \ref{richard} and \ref{shaw}, below.
\item Restructured thin plate (\pref{cor-tprs}) and cubic spline (\pref{cor-cubic}) explanation.
\item Elaboration on soap film and added stronger links to chapter 2.
\textit{Chapter 1 in general needs to be fleshed out, so as better to set the stage for the remainder of the thesis. For example, the soap film smoother is mentioned almost parenthetically at present, whereas it is adopted as a benchmark for the remainder of Part 1 of the thesis. We do not necessarily ask that all of the details are moved from Chapter 2 (where they appear at present) to Chapter 1: however, some intuition needs to be given, and the key role of this technique needs to be made clear. This would also provide the opportunity to motivate the inclusion of Chapter 2 much more convincingly: the chapter can be presented as a detailed introduction to, and illustration of, the use of the soap film smoother which then sets the stage for the subsequent development.}
\item 
\begin{itemize}
	\item \pref{cor-r22} Explained that \texttt{soap} and \texttt{mgcv} were used.
	\item \pref{cor-soft1} Clarified that all the models were fitted using existing libraries.
	\item \pref{cor-soft2}, \pref{cor-soft4} Added mention of the \textit{SC Toolbox} and its author.
	\item \pref{sc-sims} Explains that the MATLAB package \textit{SC Toolbox} is used for the \sch\ mapping and \texttt{mgcv} and \texttt{soap} are used for smoothing.
	\item \pref{cor-soft3} Clarified that I wrote the code for finding the distances and performing Gower's interpolation. \texttt{soap} and \texttt{mgcv} were used for smoothing. \textsf{R}'s built-in MDS routine was used.
	\item \pref{cor-soft5} Clarified that \texttt{mrds} and \texttt{mmds} were used to fit the case studies for the mixture models.

\end{itemize}
\textit{Throughout the thesis, we would like you to be clear about what software is being used: at present, it is not always clear whether you are picking up and running with �off-the shelf� software (which turned out to be the case for the Schwarz-Christoffel transform) or whether you had to write new code yourself. The work of others should always be duly acknowledged; and you shouldn?t be shy about advertising your own contributions! }
\item \textit{In Chapter 4, the discussion of the �adjustment� {\cal L} needs to be clarified to make clear that it is heuristic: at present, it is presented as somehow deeper than this. As discussed during the oral examination, one possibility would be to start by considering the idea of using the Jacobian of the transformation, explaining why (as indicated by the candidate during the oral) this didn?t work, and then moving on to the �engineering� solution.}
\item Chapter 7 has been restructured, this is detailed below.
\end{itemize}


\section{Minor corrections from Richard Chandler}
\label{richard}

\begin{enumerate}
\item p. 26 Changed ``this data'' to ``these data''. Also: \pref{cor-r1-2}, \pref{cor-r1-3}, \pref{cor-r1-4}, \pref{cor-r1-5} ``is'' $\rightarrow$ ``are''; \pref{cor-r1-6}, \pref{cor-r1-7} ``was'' $\rightarrow$ ``were''; \pref{cor-r1-8} ``it'' $\rightarrow$ ``they''.
\item p. 26 Changed to talk about trend surface, rather than process.
\item p. 28 ``Complementary''
\item p. 29, 30, [[TKTKTK p66:13, p227:6 find at end!]] ``effect'' $\rightarrow$ ``affect''.
\item p. 30, changed caption to say that the function begins to interpolate the data.
\item p. 30, difference between thin plate splines and thin plate regression splines given from the outset.
\item p. 38 ``datum'' $\rightarrow$ ``data''
\item p. 40. now reads as requested.
\item p. 41 ``PIRLS'' $\rightarrow$ ``IRLS'', included definition of $\mathcal{V}_g$ in the GAM case.
\item Section 1.3 title ``practice'' $\rightarrow$ ``practise'', same p. 183
\item p. 42 corrected two typos
\item (1.10) corrected Brier definition to have the data in.
\item p. 44 explained EDF--complexity link better.
\item p. 45 corrected two typos. \textbf{qualification for the ``flawed'' nature of existing models is in the following text both in section 1.4.1 and the abstract}
\item p. 49 corrected typo
\item p. 50 unsure what I meant by ``leakage as a breakdown of stationarity'', I've elaborated further on how these methods work as an introduction to the MDS stuff that comes later on.
\item p. 50 corrected typo
\item \textit{Chapter 2} \pref{cor-r18} corrected typo.
\item \pref{cor-r19} Added comment about missing data, changed plot to highlight those cells with no data to be blue.
\item \pref{cor-r20} Expanded section on Tweedie distribution.
\item \pref{cor-r21} Explained why lat/long lead to anisotropy.
\item \pref{cor-r22} Said that \texttt{soap} and \texttt{mgcv} were used to construct the model.
\item \pref{cor-r23} Added an extra line explaining that $\lambda_\text{int}$ and $\lambda_\text{bnd}$ are actually estimated rather than $\lambda_\text{space}$. I'm inclined to leave (2.3) as having only $\lambda_\text{space}$ in as it's true in general for tensor products and it makes the explanation of the tensor product simpler. What do you think?
\item \pref{cor-r24} Added assumptions for posterior.
\item \pref{cor-r25} Added Richard's paper to the bibliography.
\item \pref{cor-r26} Added figure showing spatial groups and changed caption accordingly. Caption also claimed that the grid was 10x10, it was 5x5. Added comment on the line in the scale-location plot, which does indeed correspond to zeros in the data, coloured these data grey.
\item \pref{cor-r27} Added comment about how tight the intervals in figure 2-5 are.
\item \pref{cor-r28} Added timing for model fitting.
\item \textit{Chapter 3} \pref{cor-r29} Much of this is covered in the re-write of the rectangle section (see corresponding question from SS).
\begin{itemize}
	\item \pref{cor-r29-1} Tidied up and reordered these paragraphs so that we know we're talking about complex numbers before we start.
	\item \pref{cor-r29-2} I've removed the equation for the Jacobi elliptic sine function; I don't think that laying out the definition here is actually very useful to the understanding of the \sch\ transform itself and distracts from the rest of the section. The full explanation would probably take up about another page, which seemed like a major diversion.
	\item \pref{cor-r29-3} ``can be computed''.
	\item \pref{cor-r29-4} Removed definition of $J_e$, see two above.
	\item \pref{cor-r29-5} More detail on the strip mapping added.
	\item \pref{cor-r29-6} Removed sentence, as advised by SS, below.%corrected to ``known''.
\end{itemize}
\item \textit{Chapter 4} \pref{cor-r30}, \pref{cor-r30-1} ``principle'' $\rightarrow$ ``principal''.
\item \pref{cor-r31} Clarified that we're pretending that we don't know the $x_i$s.
\begin{itemize}
	\item \pref{cor-r31-1}``The first two terms in on'' $\rightarrow$ ``The contributions from the first two terms''
	\item \pref{cor-r31-2} \textbf{Not clear what Richard means here, is there any way you can clarify?}
	\item \pref{cor-r31-3} Removed sentence as suggested.
	\item \pref{cor-r31-4} Removed paragraph as suggested.
	\item \pref{cor-r31-5-2} Corrected matrix dimensions, $\tr{(\mathbf{X}^*)} \rightarrow \tr{(\tilde{\mathbf{X}}^*)}$. \pref{cor-r31-5} $\text{diag}(\mathbb{S})_{ii} \rightarrow \text{diag}(\mathbb{S})_i$. Further clarification.
	\item \pref{cor-r31-6} Unless I'm getting the wrong end of the stick here, I think Richard's statement is incorrect. The fact that you can perform MDS on a reduced set of points, insert the other points and end up with the same configuration as if you'd done the lot from the start means that it is \textit{not} sensitive to starting values. I've clarified that this is only in the Euclidean case (according to Gower's paper anyway).
	\item \pref{cor-r31-7} ``in a similar to check'' $\rightarrow$ ``in a similar way to check''.
\end{itemize}
\item \pref{cor-r32} ``is show'' $\rightarrow$ ``is shown''.
\item \pref{cor-r33} ``novel'' $\rightarrow$ ``a novel''.
\item \pref{cor-r34}  ``the'' $\rightarrow$ ``that''.
\item \pref{cor-r35-1} Spelled out EDF, clarified the noise levels, changed ``models'' to ``fitting methods'' (also on \pref{cor-r35-2}).
\item \pref{cor-r36} Added explanation of the change in surface from the earlier peninsula domain.
\item \pref{cor-r37} ``sever'' $\rightarrow$ ``severe''
\item \pref{cor-r38} Changed axis labels to be correct.
\item \pref{cor-r39}
\item \pref{cor-r40} Noted that numerical calculation of the penalty wasn't necessary before.
\item \pref{cor-r41} Removed $\dagger$, relic of previous document.
\item \pref{cor-r42} 
\item \pref{cor-r43} Referenced back to section 1.1.
\item \pref{cor-r44} response is assumed Gamma distributed.
\item \pref{cor-r45} Corrected grammar.
\item \pref{cor-r46} Bug in plotting code caused the dimension to be shifted, corrected.
\item \pref{cor-r47} ``an as such'' $\rightarrow$ ``and as such''.
\item \pref{cor-r48} \textit{Chapter 5} ``massive'' $\rightarrow$ ``a massive''.
\item \pref{cor-r49} Removed incorrect statement about backfitting, put in argument about integrating smoothing parameter selection. Toned down kriging criticism, linked forward to full discussion in chapter 6.
\item \pref{cor-r50} ``fequencies'' $\rightarrow$ ``frequencies''.
\item \pref{cor-r51} (Assuming that GCM is a typo here and Richard means GCV). Clarified that at each stage we have GCV-optimal smoothness and that then we pick between the models using the GCV score.
\item \pref{cor-r52} Boxplots are in groups of 5!
\item \pref{cor-r53} Made this sentence make more sense.
\item I'm very much aware of red-green colourblindness as this is a condition that I suffer from! As far as I can tell, all of the figures in the thesis are visible to someone with red-green colourblindness (at least for the subtype which I have). I've tested the figure in question (5-10) and it looks fine on the website suggested.
\item \pref{cor-r55} By ``smooth'' I meant non-linear here, corrected.
\item \pref{cor-r56} \textit{Chapter 6} ``it's'' $\rightarrow$ ``its''.
\item \pref{cor-r57} ``disirable'' $\rightarrow$ ``desirable''.
\item \pref{cor-r58} Removed paragraph.
\item \pref{cor-r59} \textit{Chapter 7} Added some further discussion here. \pref{cor-r59-1} It's the pdf of the distances. Stationarity is covered in the assumptions (\pref{cor-7s2}), see also below.
\item \pref{cor-r60} Typo removed by (SS) correction, below.
\item \pref{cor-r61} Added hats to $N$, $p$ and $P_a$ as necessary. Re-write of assumptions section (see below) covers the part about the random sampling beforehand now.
\item \pref{cor-r62} Mentioned CDS/MCDS earlier.
\item \pref{cor-r63}
\item \pref{cor-r64} Typo corrected, see modifications below regarding plotting of covariates.
\item \pref{cor-r65} Plotting of covariates is covered below. Added legend to caption.
\end{enumerate}

\section{Minor corrections from Simon Shaw}
\label{shaw}

\subsection{Chapter 1}
\begin{itemize}
\item p. 29 Added $\dots$ (twice).
\item p. 29 Reference corrected.
\item p. 30 Added $\dots$, removed emboldening.
\item p. 32 Changed ``Say we put'' to jump straight into talking about the matrix.
\item p. 34 ``practise'' $\rightarrow$``practice'', throughout.
\item p. 34,35 B-spline definition now uses $x$ throughout. Summation corrected and $i$ terms removed.
\item p. 35 Objective function corrected.
\item p. 36 Removed self-reference.
\item p. 36 Corrected definition of cubic splines.
\item \textbf{p. 36 ``p36 $\beta_j$ and $\delta_j$ should be defined in relation to the form of $f(x)$''. Can you elaborate on what this means?}
\item p. 36 [[TKTKTK 37?]]  Short discussion of cyclic splines added.
\item p. 36 Changed definition to $\delta_j = \frac{\partial^2 f(x)}{\partial x^2}\vert_{x=x_j^*}$
\item p. 37 Tensor product definition now has two different size bases.
\item p. 38,39 defined $\mathbf{\hat{f}}$.
\item p. 39 ``leasst'' $\rightarrow$ ``least''.
\item p. 40 $\bm{\beta}^{[k]} \rightarrow \bm{\hat{\beta}}^{[k]}$
\item p. 40 added $\dots$
\item p. 41 smoothing parameter does need to be a vector, changed later references
\item p. 41 corrected reference
\item p. 42 added brackets to references
\item p. 45 cleared up explanation of a complex boundary with an example
\item p. 45 fixed typo
\item p. 47 fixed typo
\item p. 48 removed
\item p. 49 fixed typo
\end{itemize}

\subsection{Chapter 2}
\begin{itemize}
\item \pref{cor-2s1} Corrected.
\item \pref{cor-2s2} Removed.
\item \pref{cor-2s3} \Secref{mds-faster} deals with ensuring that the model we're using is fast enough not to cause frustration to users. Should I reference this here?
\end{itemize}

\subsection{Chapter 3}
\begin{itemize}
\item \pref{cor-3s1-1} Changed $\Gamma$ to $W$. \pref{cor-3s1-3} stated that $W$ is the inside of $\Gamma$.
\item \pref{cor-3s2} Removed references to $\varphi$ at this point.
\item \pref{cor-3s3} Expanded opening paragraph(s) to discuss the Ramsay horseshoe in a bit more depth.
\item \pref{cor-3s4} Replaced figure 3-1 with 3-5; corrected $x$ to $w$ and $w^*$ in caption; reversed diagram to be same ordering as others, later. \pref{cor-3s4-2} Added some elaboration here.
\item \pref{cor-3s5} Cleared up caption and figure. Corrected (and added) line label. Caption: $w_6^* \rightarrow w_6$. 
\item \pref{cor-3s6} Corrected.
\item \pref{cor-3s7} Cleared up caption and figure. Changed line label. ``upper half-plane'' $\rightarrow$ ``unit disc''
\item \pref{cor-3s8-1} It should be a polygon, that shouldn't have implied that wasn't the case -- corrected. \pref{cor-3s8-2} Clarified $\varphi$ and $\varphi^{-1}$ are found. \pref{cor-3s8-3} $\varphi \rightarrow \varphi^{-1}$. \pref{cor-3s8-4} Cleared up transforming back to the data domain.
\item \pref{cor-3s9} Corrected angles in diagram. Increased shading.
\item \pref{cor-3s10} Corrected as suggested.
\item \pref{cor-3s11} Final $\varphi$ should be $\varphi^{-1}$.
\item \pref{cor-3s12} Corrected ``may'' $\rightarrow $ ``many''.
\item \pref{cor-3s13-1} Attempted to clarify introduction to the section. \pref{cor-3s13-2} Clarified that the prevertices lie on the real line for the upper half plane case. \pref{cor-3s13-3} Prevertices are complex in the unit disc case. \pref{cor-3s13-4} Prevertices rectangle case are covered in re-write of that section, see below. 
\item \pref{cor-3s1-1} Changed $\Gamma$ to $W$. \pref{cor-3s14-1} First sentence the right way around. \pref{cor-3s14-2} $w_0 \rightarrow w^*_0$.  \pref{cor-3s14-3} Talking about calculation numerically, put into separate paragraph. Added some more information about the base point of the integration.
\item \pref{cor-3s15-2} Added further explanation to the rectangle map, including a figure illustrating how the series of maps work.
\item \pref{cor-3s16} Put this computational bit in a separate paragraph, cleared up who solved the numerical issue.
\item \pref{cor-3s17} Introduced the section a bit more.
\item \pref{cor-3s18-1} Removed ref to (3.3), twice. \pref{cor-3s18-2} I've reordered this paragraph, putting the rectangle first and referencing where that comes from. I've left the unit disc but in since I do show examples of the unit disc later. 
\item \pref{cor-3s19} Removed sentence, don't really need to know this.
\item \pref{cor-3s20} Added initial values in step 1 of algorithm.
\end{itemize}

\subsection{Chapter 4}
\begin{itemize}
\item \pref{cor-4s1}
\item \pref{cor-4s2} Emboldened.
\item \pref{cor-4s3} Added ``the''.
\item \pref{cor-4s4} Re-ordered figures.
\item \pref{cor-4s5} Corrected caption.
\item \pref{cor-4s6} Cleared up contradiction.
\item \pref{cor-4s7} Corrected value in table.
\item \pref{cor-4s8} Grammar correction.
\item \pref{cor-4s9} Changed equation as suggested.
\item \pref{cor-4s10} Corrected line break.
\item \pref{cor-4s11}
\end{itemize}

\subsection{Chapter 5}
\begin{itemize}
\item \pref{cor-5s1} Added full stop.
\item \pref{cor-5s2} Added ``that''.
\item \pref{cor-5s3} Corrected equation.
\item \pref{cor-5s4} Changed reference to be to equation in this chapter rather than in chapter 1, did the same for the basis function.
\item \pref{cor-5s5} Added ``for some choice of $s$''. Added some description of the role of $s$ here and re-ordered the section.
\item \pref{cor-5s6} Clarified that I mean the smallest $s$ satisfying (\ref{duchon-s-eqn}).
\item \pref{cor-5s7} Replaced hyphen with colon.
\item \pref{cor-5s8} ``$l$'' $\rightarrow$ ``$\hat{l}$''.
\item \pref{cor-5s9} Removed ``the''.
\item \pref{cor-5s10} Removed reference to section 1.2.
\item \pref{cor-5s11} Corrected grammar.
\item \pref{cor-5s12} Corrected reference. 
\item \pref{cor-5s13} $\mathbf{x}_{i} \rightarrow \mathbf{m}_{i}$
\item \pref{cor-5s14} ``MP example'' $\rightarrow$ ``MP voting data example'' 
\item \pref{cor-5s15} ``fit'' $\rightarrow$ ``fitted'' 
\end{itemize}

\subsection{Chapter 6}
\begin{itemize}
\item \pref{cor-6s1} Corrected reference.
\item \pref{cor-6s2} Corrected reference. 
\item \pref{cor-6s3}  Added ``on''.
\item \pref{cor-6s4} Added ``of''.
\end{itemize}

\subsection{Chapter 7}
\begin{itemize}
\item \pref{cor-7s1} ``can be written as'' $\rightarrow$ ``can be estimated by''.
\item \pref{cor-7s2} Re-ordered section. Removed the field procedure-based assumptions, since we're not really interested in that here. Covered the uniform density wrt to line assumption (random line placement) to begin with, as suggested later.
\item \pref{cor-7s3} ``animal'' $\rightarrow$ ``object''.
\item \pref{cor-7s4} ``figures'' $\rightarrow$ ``figure'.
\item \pref{cor-7s5} Added explanation of the ``key function plus adjustment terms'' formulation.
\item \pref{cor-7s6} Removed parentheses, corrected reference.
\item \pref{cor-7s7} Put in $x_i$.
\item \pref{cor-7s8} ``Instead'' $\rightarrow$ ``instead''.
\item \pref{cor-7s9} Removed specific example and re-ordered.
\item \pref{cor-7s10} Pointed out that the covariates only affect the scale but not shape, referred back to the detection function definitions. Added further discussion of assumptions. \pref{cor-7s10-2} Added reference to \citeb{davidbthesis}.
\item \pref{cor-7s11} Corrected definition of $\nu$/$\nu_i$.
\item \pref{cor-7s12} Added that random line/point placement is critical into ``Assumptions''.
\item \pref{cor-7s13} Put the summary at the end.
\item \pref{cor-7s14} Added a paragraph on how people deal with monotonicity. \pref{cor-7s14-2} explicitly mentioned the Distance software, then cited the standard reference for the software.
\item \pref{cor-7s15} Moved acknowledgement.
\end{itemize}

\subsection{Chapter 8}
\begin{itemize}
\item \pref{cor-8s1} ``being fit to data'' $\rightarrow$ ``fitted to the data''.
\item \pref{cor-8s2} Referenced the added detection functions.
\item \pref{cor-8s3} Corrected equation (\label{mix-detfct}). \pref{cor-8s3-2} Added further discussion in previous chapter about covariates. \pref{cor-8s3-1} Added reference to \cite{covpaper}. \pref{cor-8s3-3} Explicitly said that I'm modelling the detection function as conditional on the covariates, as advised.
\item \pref{cor-8s4} explicitly stated $\sigma_{ij}=\sigma_i\sigma_j$.
\item \pref{cor-8s5} I do mean the conditional likelihood, I've cleared up the section inline with the corrections detailed above.
\item \pref{cor-8s6} ``$\mathbf{Z}$'' $\rightarrow$ ``$\mathbf{z}_i$''. (Also corrected throughout Appendix 1.)
\item \pref{cor-8s7} ``$\mathbf{Z}$'' $\rightarrow$ ``$\mathbf{z}_i$''.
\item \pref{cor-8s8} Cleared up subscripts.
\item \pref{cor-8s9} Clarified what I mean when sorting the distances then splitting them into groups to obtain starting values.
\item \pref{cor-8s10} Detection functions now defined in the previous chapter. \pref{cor-e10} Cleared up what I mean by a ``a $J$-point mixture''.
\item \pref{cor-8s11-1} Expanded discussion of the results for the \citeb{williams} data. \pref{cor-8s11} Expanded caption in figure \ref{williams-table}. \pref{cor-8s11-3} Added some discussion of the harbour seal fits.
\item \pref{cor-8s12-1} Added a section about how the plotting is done for covariates. \pref{cor-8s12-2} Added some more clarification here.
\item \pref{cor-8s13} Added discussion of $p$-values for the amakihi data in the main body.
\end{itemize}

\section{Things I found as I went through\dots}

\begin{itemize}
\item p. 36, added limits to the penalty for cubic splines.
\item p. 26 " (in the sense that they take the value one at one knots and zero at all others)" $\rightarrow$  (in the sense that they take the value one at one knot and zero at all others)
\item p. 36 Noted that the $b_j$s that result are implicit.
\item Chapter 3 ``disk'' $\rightarrow$ ``disc''.
\item \pref{cor-e5} Full stop should have been a colon.
\item \pref{cor-e6} ``near the other vertices'' $\rightarrow$ ``near any other vertex''.
\item \pref{cor-e7} Clarified the \sch\ algorithm.
\item \pref{cor-e8} Added some brackets to ``$n-1$-dimensional'' for clarity.
\item \pref{cor-e9} Corrected bracketing for Gower reference.
\item \pref{cor-e11} ``fine'' $\rightarrow$ ``find''.
\item \pref{cor-e12} ``a'' $\rightarrow$ ``at''.
\item \pref{cor-e13} ``sate'' $\rightarrow$ ``state''.
\item \pref{cor-e14} Corrected Pike reference.
\item \pref{cor-e14} Caption claimed the Pike model was half-normal, when it in fact has a cosine adjustment.
\item \pref{cor-e15} Added some more information on habitat type.
\item \pref{cor-e16} Added level legend for Pike detection function plot.
\end{itemize}

