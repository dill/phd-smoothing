\documentclass[10pt]{article}

\usepackage{url}

\renewcommand{\title}[1]{\begin{center}{\bf \LARGE #1}\end{center}}
\newcommand{\affiliations}{\footnotesize}
\newcommand{\keywords}{\paragraph{Keywords:}}

\setlength{\oddsidemargin}{0cm} \setlength{\evensidemargin}{0cm}
\setlength{\textwidth}{16.5cm} \setlength{\topmargin}{-1cm}
\setlength{\textheight}{24.5cm}

\begin{document}
\pagestyle{empty}

\title{A domain-morphing approach to smoothing over complex regions}

\begin{center}
  {\bf David Lawrence Miller$^{1,*}$}
\end{center}

\begin{affiliations}
1. University of Bath\par
* Contact author: d.l.miller@bath.ac.uk
\end{affiliations}

\keywords Smoothing; Generalizedd Additive Models (GAMs); Spatial smoothing; penalized regression splines.

\vskip 0.8cm

Smoothing over complex 2-D regions is difficult. Several approaches have been proposed in past years including finite element analysis (Ramsay, 2002), within-area distance (Wang \& Ranalli, 2007) and recently soap film smoothing (Wood, Bravington \& Hedley, 2008.) Here I investigate two alternative methods.

The first (following from Eilers, 2006) is based on the Schwarz-Christoffel transform from complex analysis. This takes the region and ``morphs'' it to a rectangle or disk in a prescribed way. 

The second uses multidimensional scaling to create within-area distances to morph the shape into a more suitable shape.

In both cases we may then smooth over the transformed area using penalized regression splines and transform this smooth back to the original domain in order to perform analysis. I explore the utility of both of these transforms on both real and simulated data.

\paragraph{References}
\begin{description}
\item Ramsay, T. (2002) Spline smoothing over difficult regions. JRSSB, 64(2), 307-319.
\item Wang, H. and M.G. Ranalli (2007) Low-Rank Smoothing Splines on Complicated Domains. Biometrics, 63(1), 209-217.
\item Wood, S.N., M.V. Bravington and S.L. Hedley (2008) Soap film smoothing. JRSSB, 70(5), 931-955.
\item Eilers, P. H. C. (2006) P-spline smoothing on difficult domains. Seminar at the University of Munich.

\end{description}

%\nocite{ref1,ref2}
%\bibliographystyle{amsplain}
%\bibliography{biblioExample}

\end{document}
