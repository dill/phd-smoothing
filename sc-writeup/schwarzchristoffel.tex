% Brief explanation of the Schwarz-Christoffel transform
% David Lawrence Miller
% d.l.miller@bath.ac.uk
 
% Started : 29 October 2008
% Completed (first draft) : 
% Really completed :
 
\documentclass[a4paper,10pt]{amsart}
 
% Load some packages
\usepackage{times, amsmath, amssymb, amsfonts, url, natbib, graphicx, multirow, bm, rotating}
 
\usepackage{multirow}
 
% top matter
\title{The Schwarz-Christoffel transform and its possible application to finite area smoothing}
\author{David Lawrence Miller}
\email{d.l.miller@bath.ac.uk}
\address{Mathematical Sciences, University of Bath, Bath, United Kingdom}
 
% Shortcuts
% Probability
\newcommand{\prob}[1]{\mathbb{P}\left[ #1 \right]}
% Hovitz-Thompson
\newcommand{\HT}{\hat{\tau}_{HT}}
% Schwarz-Christoffel
\newcommand{\sch}{Schwarz-Christoffel }
% fprime
\newcommand{\fprime}{f^\prime(z)}

\begin{document}
 
% The abstract
\begin{abstract}
Smoothing in a complex region is difficult. Current approaches are computationally intensive. Here we propose a method for using the \sch transform to warp a region to a more managable shape in order to smooth over points in it.
\end{abstract}
 
 
% New theorem for theorems
\newtheorem{thm}{Theorem}[section]
 
%New theorem for definitions
\newtheorem{defn}{Definition}[section]
 
\maketitle
 
\section{Conformal Mappings}

In complex analysis we define a conformal mapping as an analytic, injective map from the $z$ plane to the $w$ plane (\cite{handbuch}, \emph{p. 672}), ie,

\begin{equation*}
w=f(z)=u+iv, \qquad f^\prime(z) \neq 0.
\end{equation*} 

The transformation of a line segment is the combination of a dilation (by a factor of $\sigma=\vert f^\prime(z) \vert$) and a rotation (by $\alpha = \arg f^\prime(z)$.) A conformal mapping preserves the angle of intersection of curves.

\section{\sch Mapping}

The \sch mapping (or transform or formula) is a method of conformally mapping the interior (or exterior) of a polygon to another domain (most commonly the unit disk or the upper half-plane.)

This is achieved by looking at the derivative of the conformal map $f$. For some maps we may be express the derivative as:

\begin{equation}
\fprime = \prod f_k,
\end{equation}

for some (cannonical) functions $f_k$. So we can say that:

\begin{equation}
\arg \fprime = \prod \arg f_k.
\end{equation}

In the classical mapping (ie. from polygon to the half-plane) we see that each $\arg f_k$ is designed to be a step function such that $f$ maps the real line onto the boundry of the polygon (henceforth "the polygon".) Usually with conformal mappings it is the border that causes the most problems, however, with the \sch map we can use the Schwarz relflection principle (\cite{handbuch}, \emph{p. 679}) we see that $f$ is analytic between the vertices on the polygon, $\fprime$ exists and hence $\arg \fprime$ is constant. We also know that $\arg \fprime$ undergoes a specific jump at each vertex, that being the (external) turning angle (denoted $\beta_k$ \footnote{$\beta_k = 1-\alpha_k$, $\alpha_k$ being the internal angle}) required at each vertex. 

From this we can identify some function for $f_k$ which is analytic on $H^+$\footnote{The upper half-plane.} or the unit disk and satisfies the above conditions. In the half-plane case this is:

\begin{equation}
\label{halfplanefunc}
f_k = (z-z_k)^{-\beta_k},
\end{equation}

where the $k$s index the points on $\mathbb{R}$ that map to polygon's vertices. We pick the branch such that $f_k(z)>0$ if $z>z_k$ on $\mathbb{R}$

\begin{equation}
\fprime = C \prod_{k=1}^{n-1} f_k(z).
\end{equation}

So, integrating and substituting in (\ref{halfplanefunc}) we obtain:

\begin{equation}
f(z) = A + C \int^z \prod_{k=1}^{n-1} (\zeta-z_k)^{-\beta_k} d\zeta,
\end{equation}

where $A$ and $C$ are complex constants. This is known as the \emph{\sch formula} for the half-plane.


\subsection{Nomenclature}

At this point it is useful to take note of some of the nomenclature related to \sch mappings and to address some definitions.

We first give a definition for the polygon and related quantities.

\begin{defn}[Polygon]
We define a polygon, $\Gamma$, as a collection of vertices $w_1, w_2,\dots,w_n$ and interior angles $\alpha_1\pi, \alpha_2\pi, \dots, \alpha_n\pi$. For convenience we define $w_{n+1} = w_1$ and $4_0=w_n$. The $w$s lie in $\mathbb{C} \cup {\infty}$. Numbering of vertices is counterclockwise (ie. the polygon is "to the left" as one traverses $w_k$ to $w_{k+1}$. The angles are such that $\alpha_k \in (0,2]$ and we require:

\begin{equation}
\sum_{k=1}^n (1-\alpha_k) = 2.
\end{equation}

Hence, the exterior angle, $\beta_k\pi$ is given by $(1-\alpha_k)\pi$.

Usually, $\Gamma$ is used to refer to the boundary of the polygon and $P$ to the region inside.

\end{defn}

Since in the \sch transform we are actually looking for the inverse of the map from the polygon to the plane (or disk). We use the function $f$ to go from the unit disk or $H^+$ to the polygon and $f$s inverse, $F$ to go from the polygon to the disk or half-plane.  

We need to know the position that the vertices on $\Gamma$ (the $w_k$s) correspond to on the real line (or the boundry of the unit disk). These points are referred to as \emph{prevertices} and are denoted $z_k$.

It is certainly worth noting at this point (to avoid confusion) that those points on the polygon are always referred to as $w_i$ (for some $i$) and those on the unit disk or $H_+$ as $z_i$ (for some $i$.)




\subsection{Mapping to the upper half-plane}

When we map $P$ to $H^+$ we set $f(\infty) = w_n$ without any loss of generality. We then have the formula (as above):

\begin{equation}
f(z) = A + C \int^z \prod_{k=1}^{n-1} (\zeta-z_k)^{-\beta_k} d\zeta.
\end{equation}



\subsection{Unit disk}

For the unit disk we do not fix any points, however note below that the product now runs over all of the prevertices. The integrand is simply a constant multiple of the original form. This is merely to avoid problems in the calculation of the branch cuts (\cite{driscoll}, \emph{p. 12}).

\begin{equation}
\label{unitscmap}
f(z) = A + C \int^z \prod_{k=1}^{n} (1 - \frac{\zeta}{z_k})^{-\beta_k} d\zeta.
\end{equation}


%\section{Special cases}




\section{Computation of the \sch mapping}

It is generally easier to compute the \sch map to the unit disk rather than to $H^+$. For that reason we only address that here. We now detail the process for computing the map as well as covering some of the pitfalls which may occur when computing the map.

To compute the map, we use an iterative routine to solve, numerically, the following set of equations:

\begin{equation}
\label{optimizeme}
\frac{\vert \int_{z_j}^{z_{j+1}} f^\prime(\zeta) d\zeta \vert}{\vert \int_{z_2}^{z_{1}} f^\prime(\zeta) d\zeta\vert} - \frac{\vert w_{j+1} - w_j\vert}{\vert w_2 - w_1\vert} = 0, \qquad \text{for } j=2,3,\dots,n-2,
\end{equation}

Where $f$ is as in (\ref{unitscmap}).

This set of equations should be fairly intuitive since we are just looking at the iodifferences between the actual distances between the $w$s and those given by the current set of parameters in the current approximation for $f$.

Note that the relation does not include the vertex $w_n$, nor do we look at $w_1$ or $w_2$ in the right hand side of the above equation. This is since (by theorem 3.1 of \cite{driscoll} \emph{p. 24}) a polygon is precisely defined by its angles and its vertices not including $w_1$, $w_2$ and $w_n$.

So, we proceed in estimating $A$, $C$ and $w_i$ for $i=1,\dots,n$ using a quasi-Newton solver\footnote{We actually use steepest descent in the early iterations, followed by Newton's method later.}. In order to calculate the integrals in (\ref{optimizeme}) we use the Guass-Jacobi quadrature.

In the software \texttt{SCPACK}\footnote{Available from \url{http://netlib.org/conformal/scpack} and \url{http://netlib.org/conformal/sclib}} written by Trefethen, the following steps are taken to compute the map (\cite{scdoc}).

\begin{enumerate}
\item Subroutine \texttt{QINIT} is called to compute Gauss-Jacobi quadrature.
\item Subroutine \texttt{SCSOLV} is called to compute the prevertices and complex onstants in the \sch mapping.
\item Subroutine \texttt{WSC} or \texttt{ZSC} is called to compute the forwards or backwards map for a series of points.
\end{enumerate}

We now address each of these steps in more detail.

\subsection{\texttt{QINIT}}

Here we set up the nodes and weights in the Gauss-Jacobi quadrature to evaluate the integrals. The Gauss-Jacobi quadrature is given by the following formula:

\begin{equation}
\int_{-1}^{1} (1-x)^\alpha (1+x)^\beta f(x) dx = \sum_{i=0}^{n-1}h_if(x_i) + \epsilon_n,
\end{equation}

where the $h_i$s are weights and the $x_i$ are referred to as nodes. $E_n$ is an error term. The integral is rescaled to be over the correct interval (\cite{trefethen}, \emph{p. 11}). We do not go into any further detail and treat this process as a "black box".

\subsection{\texttt{SCSOLV}}

The routine \texttt{SCSOLV} does the main work find the parameters of $f$. Using the nodes and weights for the Gauss-Jacobi quadrature found by \texttt{QINIT}. We then iterate through possible values for the prevertices in (\ref{unitscmap}) in order to minimise the result of the left hand side of (\ref{optimizeme}).

\subsection{\texttt{WSC} and \texttt{ZSC}}

\subsubsection{Forwards map}

Once the prevertices and complex constants have been calculated, the hard work is over. Calculating the forwards map (\texttt{WSC}) is simply a case of evaluating $f$ at the necessary points. For clarity if we wish to find the point on polygon ($w$) given we know the point on the disk ($z$) we compute:

\begin{equation}
\label{unitscmap}
w=f(z) = w_0 + C \int_{z_0}^{z} \prod_{k=1}^{n} (1 - \frac{\zeta}{z_k})^{-\beta_k} d\zeta,
\end{equation}

where $z_0$ is any point in the closed disk where $w_0 = f(z_0)$ is known and non-infinite. We may choose any point since the integrand is analytic throughout the mapping and hence the integral is path-independent (\cite{driscoll} \emph{p. 27}).


\subsubsection{Backwards map}

For calculating to get back from the polygon to the unit disk, we treat the equation $w(z)=w$ as a non-linear equation to be solved for $z$ (\cite{trefethen}, \emph{pp. 16-17}). These are solved iteratively as a series of first order ODEs, with the starting values coming from inverting (\ref{unitscmap}).














\bibliographystyle{plainnat}
\bibliography{sc-refs}

\end{document}
