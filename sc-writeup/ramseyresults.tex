% "Brief" explanation of the Schwarz-Christoffel transform
% David Lawrence Miller
% d.l.miller@bath.ac.uk

% Started : 29 October 2008

\documentclass[a4paper,10pt]{amsart}

% Load some packages
\usepackage{times, amsmath, amssymb, amsfonts, url, natbib, bm, rotating}

\usepackage{multirow}
\usepackage{graphicx}

% top matter
\title{a title}
\author{David Lawrence Miller}
\email{d.l.miller@bath.ac.uk}
\address{Mathematical Sciences, University of Bath, Bath, United Kingdom}

% Shortcuts
% Probability
\newcommand{\prob}[1]{\mathbb{P}\left[ #1 \right]}
% Hovitz-Thompson
\newcommand{\HT}{\hat{\tau}_{HT}}
% Schwarz-Christoffel
\newcommand{\sch}{Schwarz-Christoffel }
% fprime
\newcommand{\fprime}{f^\prime(z)}

\begin{document}

% The abstract
\begin{abstract}
blah abstract
\end{abstract}


% New theorem for theorems
\newtheorem{thm}{Theorem}[section]

%New theorem for definitions
\newtheorem{defn}{Definition}[section]

\maketitle



\section{Ramsey's horseshoe}

Different approaches: Ramsey, Wang+Ranalli, soap

Write intro para...

What is a complex region?



In order to deal with the problems of a complex region, so far, 4 different approaches have been proposed:

\begin{enumerate}
\item \cite{ramsay} proposes

gradient normal to the surface stuff



\item \cite{wangranalli} adopt a ``within-area distance'' formulation for thin plate splines.




\item \cite{soap} use the 

\item An alternative approach is to ``morph'' the area in question to be one that is more suitable for smoothing upon.

\end{enumerate}




\section{Method}

Here we formulate a conformal mapping from the domain of the problem (we call this the $W$ domain) to a domain on which it is easier to smooth (and where we can avoid the problems detailed above.)

\subsection{\sch}


\subsection{Problems}


\subsection{Procedure}

Our procedure consists of the following steps:

Determine the area over which we would like to smooth, $W$.

Compute the \sch transform of $W$ to get $Z$.

Map the co-ordinates of the datapoints in $W$ to $Z$.

Fit the GAM to the data in $Z$.







Brief bit about SC.

Algorithm that we use.

Matlab+R approach.



\section{Simulations}

What do we want to look at?

Why p-splines?


\section{Analysis}


Linear model stuff, will this work elsewhere?














\bibliographystyle{plainnat}
\bibliography{sc-refs}



\end{document}
