% "Brief" explanation of the Schwarz-Christoffel transform
% David Lawrence Miller
% d.l.miller@bath.ac.uk

% Started : 29 October 2008

\documentclass[a4paper,10pt]{amsart}

% Load some packages
\usepackage{times, amsmath, amssymb, amsfonts, url, natbib, bm, rotating}

\usepackage{multirow}
\usepackage{graphicx}

% top matter
\title{a title}
\author{David Lawrence Miller}
\email{d.l.miller@bath.ac.uk}
\address{Mathematical Sciences, University of Bath, Bath, United Kingdom}

% Shortcuts
% Probability
\newcommand{\prob}[1]{\mathbb{P}\left[ #1 \right]}
% Hovitz-Thompson
\newcommand{\HT}{\hat{\tau}_{HT}}
% Schwarz-Christoffel
\newcommand{\sch}{Schwarz-Christoffel }
% fprime
\newcommand{\fprime}{f^\prime(z)}

\begin{document}

% The abstract
\begin{abstract}
blah abstract
\end{abstract}


% New theorem for theorems
\newtheorem{thm}{Theorem}[section]

%New theorem for definitions
\newtheorem{defn}{Definition}[section]

\maketitle



\section{Ramsey's horseshoe}

Different approaches: Ramsey, Wang+Ranalli, soap





\section{Method}

Brief bit about SC.

Algorithm that we use.

Matlab+R approach.



\section{Simulations}

What do we want to look at?

Why p-splines?


\section{Analysis}


Linear model stuff, will this work elsewhere?














\bibliographystyle{plainnat}
\bibliography{sc-refs}



\end{document}
